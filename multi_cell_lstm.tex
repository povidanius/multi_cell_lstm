\documentclass[a4paper,11pt]{article}
 
\usepackage[english]{babel}
\usepackage[T1]{fontenc}
\usepackage[ansinew]{inputenc}
\usepackage{bbold}
\usepackage{bold-extra}
\usepackage{lmodern}
\usepackage{amsmath}
\usepackage{amsthm}
\usepackage{amsfonts}
\usepackage{gensymb}
\usepackage{mathtools}
\usepackage{subfigure}
\usepackage{theorem}
\usepackage{bm}
\usepackage{xcolor}
\usepackage[unicode]{hyperref}
\usepackage{algorithm}
\usepackage{algorithmic}
\usepackage{tikz}
\usepackage{graphicx}
\usepackage{lipsum}
\usetikzlibrary{positioning}
\usepackage{tikz}
\usepackage{empheq}
\usepackage{booktabs}
\usepackage{authblk}

%\usepackage{3dplot} %requires 3dplot.sty to be in same directory, or in your LaTeX installation
%\usepackage{amsmath,amsfonts,amssymb,amsthm,epsfig,epstopdf,titling,url,array}
\usepackage{xstring}

%\theoremstyle{definition}
%\newtheorem{defn}{Definition}[section]
%\newtheorem{conj}{Conjecture}[section]
%\newtheorem{exmp}{Example}[section]
\makeatletter
\newcommand{\change@uppercase@math}{%
  \count@=`\A
  \loop
    \mathcode\count@\count@
    \ifnum\count@<`\Z
    \advance\count@\@ne
  \repeat}

\newcommand{\LSTM}[1]{
  \mathrm{LSTM}(
  %(\begingroup\change@uppercase@math#1\endgroup)
}

\newcommand{\UPDATE}[1]{
  \mathrm{UPDATE}(
  %(\begingroup\change@uppercase@math#1\endgroup)
}

\newcommand{\READ}[1]{
  \mathrm{READ}(
 % (\begingroup\change@uppercase@math#1\endgroup)
}

\newcommand{\ADD}[1]{
  \mathrm{ADD}(
 % (\begingroup\change@uppercase@math#1\endgroup)
}
\newcommand{\MUL}[1]{
  \mathrm{MUL}(
 % (\begingroup\change@uppercase@math#1\endgroup)
}

\newcommand{\CIRC}[1]{
  \mathrm{circ}(
 % (\begingroup\change@uppercase@math#1\endgroup)
}

\newcommand{\ReLU}[1]{
  \mathrm{ReLU}(
 % (\begingroup\change@uppercase@math#1\endgroup)
}


\newcommand{\softmax}[1]{
  \mathrm{softmax}(
 % (\begingroup\change@uppercase@math#1\endgroup)
}


\makeatother

\newcommand*\GetListMember[2]{\StrBetween[#2,\number\numexpr#2+1]{,#1,},,\par}%
\newcommand{\tikzmark}[1]{\tikz[overlay,remember picture] \node (#1) {};}

\def\spvec#1{\left(\vcenter{\halign{\hfil$##$\hfil\cr \spvecA#1;;}}\right)}
\def\spvecA#1;{\if;#1;\else #1\cr \expandafter \spvecA \fi}

\newlength{\MidRadius}
\newcommand*{\CircularSequence}[3]{%
    % #1 = outer circle radius
    % #2 = inner circle radius
    % #3 = seqeunce
    \StrCount{#3}{,}[\NumberOfElements]
    \pgfmathsetmacro{\AngleSep}{360/(\NumberOfElements+1)}
    \pgfmathsetlength{\MidRadius}{(#1+#2)/2}
    \draw [red,  ultra thick] circle (#2);
    \draw [blue, ultra thick] circle (#1);
%    \draw [thick,->] (0, 0) -- (1.0, 0);
%    \draw [thick,->] (0, 0) -- (0.0, 1.0);
    \foreach [count = \Count] \Angle in {0,\AngleSep,..., 360} {%
        \draw [gray, ultra thick] (\Angle:#2) -- (\Angle:#1);
        \pgfmathsetmacro{\MidPoint}{\Angle+\AngleSep/2}
        \node at (\MidPoint:\MidRadius) {\GetListMember{#3}{\Count}};
    }%7
}%


\author{Povilas Daniu\v{s}is}
\author[1]{Povilas Daniu\v{s}is \thanks{povilas.daniusis@gmail.com}}


\title{Multi cell LSTM}

\begin{document}
\maketitle
\section{Introduction}

In this document we formulate new extension of long short-term memory (LSTM) recurrent neural network. 

\subsection{Conventional LSTM}
Let $x_{t} \in \mathbb{R}^{D_{x}}$ be input vectors. LSTM state $(c_{t}, h_{t})^{T}$ is defined by:

\begin{align}
\begin{split}
\label{LSTM}
i_{t} &=\sigma(W^{i}x_{t} + U^{i}h_{t-1} + b^{i})\\
f_{t} &=\sigma(W^{f}x_{t} + U^{f}h_{t-1} + b^{f})\\
o_{t} &=\sigma(W^{o}x_{t} + U^{o}h_{t-1} + b^{o})\\
\tilde{c}_{t} &= \tanh(W^{c}x_{t} + U^{c}h_{t-1} + b^{c})\\
c_{t} &=f_{t} \bullet c_{t-1} + i_{t} \bullet \tilde{c}_{t}\\
h_{t} &=o_{t} \bullet \tanh(c_{t}), \qedhere
\end{split},
\end{align}

Where $i_{t}$, $f_{t}$, $o_{t}$ are calles input, forget and output gates, $\tilde{c}_{t}$ - candidate cell vector, $c_{t}$ - cell vector, and $h_{t}$ - hidden state vector. All aforementioned variables are $D_{h}$- dimensional. By $\bullet$ we denote element-wise multiplication. Parameter count of LSTM is:
\begin{equation}
\label{eq:lstm_parameter_count}
N =  4 \cdot D_{h} \cdot (D_{x}  + D_{h} + 1),
\end{equation}
and state is defined by $2 D_{h}$ variables.
$c_{t}$ may be interpreted as internal memory of LSTM, while $h_{t}$ - represent its content, exposed by the output gate.  

\subsection{Multi cell LSTM} 
The internal memory of LSTM, $c_{t}$, shares the same dimension with $h_{t}$. However, according to our intuition $c_{t}$ should be able to store more information than $h_{t}$. We hypothesise, that extension of $c_{t}$ may increase effectiveness of LSTM approach. We suggest multi-cell variant of LSTM (MCLSTM) with $D_{p}$ cells, assembled into $D_{h} 
\times D_{p}$ matrix $C_{t}$. Internal attention $p_{t}$ controls importance weights of individual cells (columns of $C_{t}$).

\begin{align}
\begin{split}
\label{LSTM}
i_{t} &=\sigma(W^{i}x_{t} + U^{i}h_{t-1} + b^{i}) \in \mathbb{R}^{D_{h}}\\
f_{t} &=\sigma(W^{f}x_{t} + U^{f}h_{t-1} + b^{f})\in \mathbb{R}^{D_{h}}\\
o_{t} &=\sigma(W^{o}x_{t} + U^{o}h_{t-1} + b^{o})\in \mathbb{R}^{D_{h}}\\
p_{t} &=\softmax(W^{p}x_{t} + U^{p}h_{t-1} + b^{p}) \in \mathbb{R}^{D_{p}}\\
\tilde{C}_{t} &= \tanh(W^{c}x_{t} + U^{c}h_{t-1} + b^{c})1^{T}\in \mathbb{R}^{D_{h}
\times D_{p}}\\
C_{t} &=(f_{t} p_{t}^{T}) \bullet C_{t-1} + (i_{t} p_{t}^{T}) \bullet \tilde{C}_{t}\in \mathbb{R}^{D_{h}\times D_{p}} \\
h_{t} &= \frac{1}{D_{p}} o_{t} \bullet (\tanh(C_{t}) 1) \in \mathbb{R}^{D_{h}}\qedhere
\end{split},
\end{align}

Parameter count of multi cell LSTM is:
\begin{equation}
\label{eq:multi_cell_lstm_parameter_count}
N =  (4 \cdot D_{h}  + D_{p})\cdot (D_{x}  + D_{h} + 1),
\end{equation}

and state is defined by $D_{h}(D_{p} + 1)$ variables.

The implementation of MCLSTM can be downloaded from:
\url{https://github.com/povidanius/multi_cell_lstm}

\subsection{Kronecker LSTM}
\begin{align}
\begin{split}
(I_{t},F_{t}, O_{t}) &=\sigma(X_{t} \otimes  W^{i,f,o} + H_{t-1} \otimes U^{i,f,o}  +  B^{i,f,o})\\
\tilde{C}_{t} &=\tanh(X_{t} \otimes  W^{c} + H_{t-1} \otimes U^{c}  +  B^{c})\\
C_{t} &= F_{t} \bullet C_{t-1} + I_{t} \bullet \tilde{C}_{t}\\
H_{t} &= P(O_{t}\bullet \tanh(C_{t}))Q,
\end{split}.
\end{align}

where $X_{t}: D_{1}^{X} \times D_{2}^{X}$, $W: D_{1}^{W} \times D_{2}^{W}$,$H_{t}: D_{1}^{H} \times D_{2}^{H}$,
$$ IFOC: D_{1}^{X}D_{1}^{W} \times D_{2}^{X}D_{2}^{W} = D_{1}^{H}D_{1}^{U} \times D_{2}^{H}D_{2}^{U}  $$.
\end{document}


%\section{Conclusions}

%The implementation of MCLSTM can be downloaded from:
%\url{https://git hub.com/povidanius/multi_cell_lstm}

%\begin{thebibliography}{1}


%\end{thebibliography}



\end{document}